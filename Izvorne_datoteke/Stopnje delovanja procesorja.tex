\documentclass[12pt,a4paper]{article}
\usepackage[slovene]{babel}
\usepackage{fancyhdr}
\usepackage{ccicons}
\usepackage{hyperref}

\title{Stopnje delovanja procesorja (dejavnost)}
\author{Primož Lah}
\date{December 2025}

\begin{document}
\pagestyle{fancy}
\renewcommand{\headrulewidth}{0pt}
\renewcommand{\footrulewidth}{0.4pt}
\fancyhead{}
\fancyfoot[C]{
Stopnje delovanja procesorja (dejavnost), 2025, by P. Lah is licensed under \href{https://creativecommons.org/licenses/by-sa/4.0/}{CC BY-SA 4.0} \ccbysa \\
\thepage
}
\maketitle
\newpage
\tableofcontents
\newpage

\section{Uvod}
V tej datoteki so predstavljena navodila za izvedbo učne dejavnosti za predstavitev stopenj (\textit{fetch, decode, execute}) delovanja centralne procesne enote (\textit{CPE}) in cevovoda (\textit{pipeline}).
Priloženi so tudi potrebni pripomočki, opisani kasneje.
Dejavnost je priporočena za učence zadnjega triletja osnovne šole, ki že imajo določena predznanja (opisano v 1. poglavju).

\section{Predznanje}
\subsection{Potrebno predznanje}
Pred izvedbo te dejavnosti morajo učenci že poznati:
\begin{itemize}
    \item von Neumannovo arhitekturo
    \item naključni dostop
    \item delovanje pomnilnika
    \item pomnilniški naslov in naslavljanje
    \item Strojna koda (assembler)
\end{itemize}
\subsection{Priporočeno predznanje}
Za lažjo in bolj podrobno izvedbo je priporočeno, da učenci že poznajo naslednja področja:
\begin{itemize}
    \item pomnilniška hierarhija
    \item sklad
    \item osnovna zgradba CPE
\end{itemize}
\newpage

\section{Pripomočki}
Za izvedbo dejavnosti so potrebni sledeči pripomočki:
\begin{itemize}
    \item Tabla, ki se uporabi za:
    \begin{itemize}
        \item programski števec
        \item ukazni register
        \item akumulator
    \end{itemize}
    \item List z razpredelnico
    \begin{itemize}
        \item predstavlja vsebino pomnilnika in
        \item ima že vpisan potreben program
    \end{itemize}
\end{itemize}
Priloženi so primeri natisljivih pripomočkov.

\section{Izvedba}
Priporočeno minimalno število učencev za izvedbo dejavnosti je X, lahko pa se dejavnost prilagodi glede na velikost skupine.
Z večjo skupino se lahko hipotetični procesor bolj razveji, s tem da dodamo še druge komponente, ki so v opisani izvedbi opuščene (npr. registri, predpomnilniki, itd.).
Opisana je osnovna verzija, ki je zadostna za prikaz stopenj delovanja CPE in cevovoda.
\subsection{Razdelitev nalog med učenci}
    \begin{itemize}
        \item Aritmetično-logična enota (ALE)
        \item Pomnilnik (RAM)
        \item Kontrolna enota
        \item Vodilo
        \item Programski števec (PC)
        \item Ukazni register
        \item Akumulator
    \end{itemize}

\subsection{Potek dejavnosti}
Učitelj predstavlja vlogo ure - določi signal, ki šteje za novo urino periodo.
Učenca A, B in C v tem primeru predstavljajo vodila.
Po potrebi se lahko to število zmanjša in lahko en sam učenec (ali dva) prevzame delo vseh vodil.
V tem primeru vsaka oštevilčena alineja predstavlja eno urino periodo.
\begin{enumerate}
    \item   \begin{itemize}
                \item Programski števec je nastavljen na 0
                \item Učenec A pridobi število od PC in ga prenese do RAM
            \end{itemize}
    \item   \begin{itemize}
                \item RAM poišče naslov 0 v razpredelnici in učencu B pove ukaz, zapisan na tem naslovu
                \item Učenec B ta ukaz prenese do ukaznega registra
            \end{itemize}
    \item   \begin{itemize}
                \item Kontrolna enota prebere ukaz iz ukaznega registra
                \item Kontrolna enota preda ukaz učencu C
            \end{itemize}
    \item   \begin{itemize}
                \item Učenec C izvede ukaz
            \end{itemize}
    \item   \begin{itemize}
                \item Učenec C zapiše rezultat v akumulator
                \item Programski števec se poveča za 1
                \item Cikel se ponovi, zdaj ima programski števec vrednost za 1 večjo od prejšnjega koraka
            \end{itemize}
\end{enumerate}

\subsection{Izvedba ukazov}
\begin{itemize}
    \item LOAD 6 - učenec od RAM pridobi vrednost na naslovu 6, v naslednjem koraku (zapis v register) to vrednost zapiše v akumulator
    \item ADD 7 - učenec od RAM pridobi vrednost na naslovu 7 in izračuna vsoto te vrednosti in vrednosti iz akumulatorja, v naslednjem koraku to vsoto zapiše v akumulator
    \item STORE 6 - učenec prebere vrednost iz akumulatorja in jo prenese do RAM, ki jo v naslednjem koraku zapiše pod naslov 6 (\emph{Pomembno: tudi če je pod naslovom 6 že neka vrednost, jo RAM pobriše in prekrije z novo vrednostjo})
    \item JUMP 1 - učenec programskemu števcu določi, naj se nastavi na vrednost 1, kar bo PC storil v naslednjem koraku, torrej naslednji cikel se začne s PC vrednostjo 1
\end{itemize}

\subsection{Variacije}
Kot že omenjeno, se lahko dejavnost izvede z manj učenci, če več funkcij dodelimo enemu učencu (to najbolj enostavno storimo pri vodilih) ali z več učenci, če dodamo še druge funkcionalnosti (npr. bolj obsežen program, predpomnilniki, ipd.).
Na začetku za enostavnost razlage lahko uro tudi izpustimo in jim predstavimo le korake, ki jih posamezno izvedejo.
Kasneje pa lahko dodamo še uro, to pa še naprej nadgradimo s hkratnim izvajanjem, s čimer uvedemo tudi idejo cevovoda.
Pri vpeljavi cevovoda je pomembno še, da opozorimo na trajanje posameznega koraka in omejitev sposobnosti cevovoda pri razlikah v trajanju. 

\end{document}