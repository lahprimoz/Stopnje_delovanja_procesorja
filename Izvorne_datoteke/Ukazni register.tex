\documentclass[a4paper]{article}
\usepackage[slovene]{babel}
\usepackage{fancyhdr}
\usepackage{ccicons}
\usepackage[hidelinks]{hyperref}

\title{Ukazni register}
\author{Primož Lah}
\date{December 2025}

\begin{document}
\pagestyle{fancy}
\renewcommand{\headrulewidth}{0pt}
\renewcommand{\footrulewidth}{0.4pt}
\fancyhead{}
\fancyfoot[C]{
Ukazni register, 2025, by P. Lah is licensed under \href{https://creativecommons.org/licenses/by-sa/4.0/}{CC BY-SA 4.0} \ccbysa \\
\thepage
}
\maketitle
Navodila za učenca, ki predstavlja kontrolno enoto.
Tu poišče navodila za izvedbo posameznega ukaza, ki jih  preda naslednjemu učencu, ki bo te korake izvedel.
\newpage

\section{LOAD $x$}
\begin{itemize}
    \item Pojdi do RAM in pridobi vrednost, ki je napisana na naslovu $x$
    \item To vrednost zapiši v akumulator
\end{itemize}

\section{ADD $x$}
\begin{itemize}
    \item Pojdi do RAM in pridobi vrednost, ki je napisana na naslovu $x$. 
    Seštej $x$ in trenutno vrednost akumulatorja
    \item To vsoto zapiši v akumulator
\end{itemize}

\section{STORE $x$}
\begin{itemize}
    \item Pojdi do akumulatorja in pridobi vrednost, ki je v njem zapisana
    \item Pojdi do RAM in zapiši to vrednost na naslov $x$
\end{itemize}

\section{JUMP $x$}
\begin{itemize}
    \item Pojdi do PC
    \item Nastavi vrednost PC na $x$
\end{itemize}

\end{document}